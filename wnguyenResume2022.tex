\documentclass{ResumeFormat} % Use the custom resume.cls style

\usepackage[left=0.5 in,top=0.5in,right=0.5in,bottom=0.5in]{geometry} % Document margins
\newcommand{\tab}[1]{\hspace{.2667\textwidth}\rlap{#1}} 
\newcommand{\itab}[1]{\hspace{0em}\rlap{#1}}
\name{William Nguyen} % Your name
% You can merge both of these into a single line, if you do not have a website.
\address{+1(714)406-8131}
\address{\href{mailto:williamngoc93@hotmail.com}{williamngoc93@hotmail.com} \\ \href{https://linkedin.com/in/william-nguyen-934552194}{linkedin.com/in/william-nguyen-934552194} \\ \href{https://captnw.github.io/}{captnw.github.io}}  %

\begin{document}

%----------------------------------------------------------------------------------------
%----------------------------------------------------------------------------------------
%	EDUCATION SECTION
%----------------------------------------------------------------------------------------
\color{orange}
\begin{rSection}{Education}
{\bf BS in Computer Science}, University of California, Irvine \hfill {\emph{Graduated June 2022}}\\
{\bf BS in Computer Game Science}, University of California, Irvine \hfill \\
GPA: 3.84
% Some other education accredition example... \\

\end{rSection}

%----------------------------------------------------------------------------------------
% TECHINICAL STRENGTHS	
%----------------------------------------------------------------------------------------
\begin{rSection}{SKILLS}
\begin{tabular}{ @{} >{\bfseries}l @{\hspace{6ex}} l }
Languages & Javascript, Python, HTML, CSS, Java, C\#, C, C++, SQL \\
Frameworks & React, React native, Node.js, Spring boot\\
Tools & Git, Visual Studio Code, Visual Studio, Unity \\ %\\
Systems & Windows (10,7), Linux (Ubuntu)
%Soft Skills & Time Management, Teamwork, Communication, Problem Solving \\
\end{tabular}\\
\end{rSection}

%----------------------------------------------------------------------------------------
% Projects
%----------------------------------------------------------------------------------------
% \item Created a XYZ feature to accomplish ABC.
%     \item Retrieved data from XYZ to for ABC.
%    \item Implemented XYZ library for ABC.
%    \item Utilized XYZ that increased A by B\%.

\begin{rSection}{PROJECTS}
\vspace{-1em}

\item \textbf{Fabflix} {} \hfill {\emph{May 2022}} % \href{www.github.com/GITHUBURL}{Text here}

% front-end 
% back-end (user id management, movie data querying / retrieval, load balancer)
Mock movie storefront where users can search and purchase digital copies of movies.

Libraries/tools used: Java, SQL, Javascript, ReactJS, React native, Stripe, Spring boot
\begin{itemize}
    \itemsep -3pt {}
     \item Set up and developed the Fabflix back end via the process of Test Driven Development.
     \item Performs 2-5 queries in the backend per search to populate movie information in the frontend.
     \item Utilized the Stripe credit card vendor API to imitate payment functionality.
 \end{itemize}

 \item \textbf{Object recognition project} {} \hfill {\emph{March 2022}}

Object detection software that can be trained with images to detect and match objects in other images.

Library/tools used: Python, Numpy, Matplotlib, SciPy, Juptyer Notebook
\begin{itemize}
    \itemsep -3pt {} 
     \item Implemented object recognition with image processing, primarily using HOG (histogram of oriented gradients).
     \item Create a template for the software to match by passing in positive and negative training images.
 \end{itemize}

 \item \textbf{Checkers AI} {} \hfill {\emph{December 2020}}

This AI simulates and backtracks via the use of search trees to play checkers.

Libraries/tools used: C++, Cmake
\begin{itemize}
    \itemsep -3pt {}
     \item Utilized Monte Carlo tree search, and backtracking to empower the checkers AI to make good moves.
     \item Improved AI's effectiveness by increasing its simulations per turn from 80 simulations to 1000 simulations.
 \end{itemize}

\item \textbf{DiscordActivityBot} {} \hfill {\emph{October 2020}}

Discord Bot to track and notify user of their activities and the server’s activity.

Libraries/tools used: Discord.py, Matplotlib, SQLite, APScheduler, pytz, asyncio.io
\begin{itemize}
    \itemsep -3pt {}
     \item Co-developed the bot alongside a fellow student and hosted the bot on AWS for 5-6 months.
     \item Deployed a Discord bot that scraped fellow server occupants’ online activities in a server of 30-40 people.
     \item Stored hashed data to SQLite database which allowed the bot to retrieve the users' activity later on.
 \end{itemize}
\end{rSection} 

%----------------------------------------------------------------------------------------
%\begin{rSection}{Extra-Curricular Activities} 
%\begin{itemize}
%    \item 	Sample bullet point.
%    \item	Sample bullet point.
%\end{itemize}


%\end{rSection}

%------------------------
% Use this more detailed section if you have Relevant work experience
% keep your resume to 1 page, if you need to remove a project to display relevant experience
% that is okay
% ----------------------------
% \begin{rSection}{EXPERIENCE}

% \textbf{Role Name} \hfill Jan 2017 - Jan 2019\\
% Company Name \hfill \textit{San Francisco, CA}
%  \begin{itemize}
%     \itemsep -3pt {} 
%      \item Achieved X\% growth for XYZ using A, B, and C skills.
%      \item Led XYZ which led to X\% of improvement in ABC
%     \item Developed XYZ that did A, B, and C using X, Y, and Z. 
%  \end{itemize}
 
% \textbf{Role Name} \hfill Jan 2017 - Jan 2019\\
% Company Name \hfill \textit{San Francisco, CA}
%  \begin{itemize}
%     \itemsep -3pt {} 
%      \item Achieved X\% growth for XYZ using A, B, and C skills.
%      \item Led XYZ which led to X\% of improvement in ABC
%     \item Developed XYZ that did A, B, and C using X, Y, and Z. 
%  \end{itemize}

% \end{rSection} 

%\begin{rSection}{Work History}
%\vspace{-1.25em}
%\item \textbf{Job Title} {Company} \hfill Month Year - Month Year
%\item \textbf{Job Title} {Company} \hfill Month Year - Month Year
%\item \textbf{Job Title} {Company} \hfill Month Year - Month Year
%\end{rSection} 

%----------------------------------------------------------------------------------------

\end{document}